%%%%%%%%%%%%%%%%%%%%%%%%%%%%%%%%%%%%%%%%%%%%%%%%%%%%%%%%%%%%%%%%%%%%%%%
%%%%%%%%%%%%%%%%%%%%%% Simple LaTeX CV Template %%%%%%%%%%%%%%%%%%%%%%%%
%%%%%%%%%%%%%%%%%%%%%%%%%%%%%%%%%%%%%%%%%%%%%%%%%%%%%%%%%%%%%%%%%%%%%%%%

%%%%%%%%%%%%%%%%%%%%%%%%%%%%%%%%%%%%%%%%%%%%%%%%%%%%%%%%%%%%%%%%%%%%%%%%
%% NOTE: If you find that it says                                     %%
%%                                                                    %%
%%                           1 of ??                                  %%
%%                                                                    %%
%% at the bottom of your first page, this means that the AUX file     %%
%% was not available when you ran LaTeX on this source. Simply RERUN  %%
%% LaTeX to get the ``??'' replaced with the number of the last page  %%
%% of the document. The AUX file will be generated on the first run   %%
%% of LaTeX and used on the second run to fill in all of the          %%
%% references.                                                        %%
%%%%%%%%%%%%%%%%%%%%%%%%%%%%%%%%%%%%%%%%%%%%%%%%%%%%%%%%%%%%%%%%%%%%%%%%

%%%%%%%%%%%%%%%%%%%%%%%%%%%% Document Setup %%%%%%%%%%%%%%%%%%%%%%%%%%%%

% Don't like 10pt? Try 11pt or 12pt
\documentclass[10pt]{article}

% This is a helpful package that puts math inside length specifications
\usepackage{calc}


% Simpler bibsection for CV sections
% (thanks to natbib for inspiration)
\makeatletter
\newlength{\bibhang}
\setlength{\bibhang}{1em}
\newlength{\bibsep}
 {\@listi \global\bibsep\itemsep \global\advance\bibsep by\parsep}
\newenvironment{bibsection}%
        {\vspace{-\baselineskip}\begin{list}{}{%
       \setlength{\leftmargin}{\bibhang}%
       \setlength{\itemindent}{-\leftmargin}%
       \setlength{\itemsep}{\bibsep}%
       \setlength{\parsep}{\z@}%
        \setlength{\partopsep}{0pt}%
        \setlength{\topsep}{0pt}}}
        {\end{list}\vspace{-.6\baselineskip}}
\makeatother

% Layout: Puts the section titles on left side of page
\reversemarginpar

%
%         PAPER SIZE, PAGE NUMBER, AND DOCUMENT LAYOUT NOTES:
%
% The next \usepackage line changes the layout for CV style section
% headings as marginal notes. It also sets up the paper size as either
% letter or A4. By default, letter was used. If A4 paper is desired,
% comment out the letterpaper lines and uncomment the a4paper lines.
%
% As you can see, the margin widths and section title widths can be
% easily adjusted.
%
% ALSO: Notice that the includefoot option can be commented OUT in order
% to put the PAGE NUMBER *IN* the bottom margin. This will make the
% effective text area larger.
%
% IF YOU WISH TO REMOVE THE ``of LASTPAGE'' next to each page number,
% see the note about the +LP and -LP lines below. Comment out the +LP
% and uncomment the -LP.
%
% IF YOU WISH TO REMOVE PAGE NUMBERS, be sure that the includefoot line
% is uncommented and ALSO uncomment the \pagestyle{empty} a few lines
% below.
%

%% Use these lines for letter-sized paper
\usepackage[paper=letterpaper,
            includefoot, % Uncomment to put page number above margin
            marginparwidth=1.2in,     % Length of section titles
            marginparsep=.05in,       % Space between titles and text
            margin=0.8in,               % 1 inch margins
            includemp]{geometry}

%% Use these lines for A4-sized paper
%\usepackage[paper=a4paper,
%            %includefoot, % Uncomment to put page number above margin
%            marginparwidth=30.5mm,    % Length of section titles
%            marginparsep=1.5mm,       % Space between titles and text
%            margin=25mm,              % 25mm margins
%            includemp]{geometry}

%% More layout: Get rid of indenting throughout entire document
\setlength{\parindent}{0in}

%% This gives us fun enumeration environments. compactitem will be nice.
\usepackage{paralist}

%% Reference the last page in the page number
%
% NOTE: comment the +LP line and uncomment the -LP line to have page
%       numbers without the ``of ##'' last page reference)
%
% NOTE: uncomment the \pagestyle{empty} line to get rid of all page
%       numbers (make sure includefoot is commented out above)
%
\usepackage{fancyhdr,lastpage}
\pagestyle{fancy}
\pagestyle{empty}      % Uncomment this to get rid of page numbers
\fancyhf{}\renewcommand{\headrulewidth}{0pt}
\fancyfootoffset{\marginparsep+\marginparwidth}
\newlength{\footpageshift}
\setlength{\footpageshift}
          {0.5\textwidth+0.5\marginparsep+0.5\marginparwidth-2in}
\lfoot{\hspace{\footpageshift}%
       \parbox{4in}{\, \hfill %
                    \arabic{page} of \protect\pageref*{LastPage} % +LP
%                    \arabic{page}                               % -LP
                    \hfill \,}}

% Finally, give us PDF bookmarks
\usepackage{color,hyperref}
\definecolor{darkblue}{rgb}{0.0,0.6,0.6}
\hypersetup{colorlinks,breaklinks,
            linkcolor=darkblue,urlcolor=darkblue,
            anchorcolor=darkblue,citecolor=darkblue}

%%%%%%%%%%%%%%%%%%%%%%%% End Document Setup %%%%%%%%%%%%%%%%%%%%%%%%%%%%


%%%%%%%%%%%%%%%%%%%%%%%%%%% Helper Commands %%%%%%%%%%%%%%%%%%%%%%%%%%%%

% The title (name) with a horizontal rule under it
%
% Usage: \makeheading{name}
%
% Place at top of document. It should be the first thing.
\newcommand{\makeheading}[1]%
        {\hspace*{-\marginparsep minus \marginparwidth}%
         \begin{minipage}[t]{\textwidth+\marginparwidth+\marginparsep}%
                {\large \bfseries #1}\\[-0.15\baselineskip]%
                 \rule{\columnwidth}{1pt}%
         \end{minipage}}

% The section headings
%
% Usage: \section{section name}
%
% Follow this section IMMEDIATELY with the first line of the section
% text. Do not put whitespace in between. That is, do this:
%
%       \section{My Information}
%       Here is my information.
%
% and NOT this:
%
%       \section{My Information}
%
%       Here is my information.
%
% Otherwise the top of the section header will not line up with the top
% of the section. Of course, using a single comment character (%) on
% empty lines allows for the function of the first example with the
% readability of the second example.
\renewcommand{\section}[2]%
        {\pagebreak[2]\vspace{1.3\baselineskip}%
         \phantomsection\addcontentsline{toc}{section}{#1}%
         \hspace{0in}%
         \marginpar{
         \raggedright \scshape #1}#2}

% An itemize-style list with lots of space between items
\newenvironment{outerlist}[1][\enskip\textbullet]%
        {\begin{itemize}[#1]}{\end{itemize}%
         \vspace{-.6\baselineskip}}

% An environment IDENTICAL to outerlist that has better pre-list spacing
% when used as the first thing in a \section
\newenvironment{lonelist}[1][\enskip\textbullet]%
        {\vspace{-\baselineskip}\begin{list}{#1}{%
        \setlength{\partopsep}{0pt}%
        \setlength{\topsep}{0pt}}}
        {\end{list}\vspace{-.6\baselineskip}}

% An itemize-style list with little space between items
\newenvironment{innerlist}[1][\enskip\textbullet]%
        {\begin{compactitem}[#1]}{\end{compactitem}}

% An environment IDENTICAL to innerlist that has better pre-list spacing
% when used as the first thing in a \section
\newenvironment{loneinnerlist}[1][\enskip\textbullet]%
        {\vspace{-\baselineskip}\begin{compactitem}[#1]}
        {\end{compactitem}\vspace{-.6\baselineskip}}

% To add some paragraph space between lines.
% This also tells LaTeX to preferably break a page on one of these gaps
% if there is a needed pagebreak nearby.
\newcommand{\blankline}{\quad\pagebreak[2]}

% Uses hyperref to link DOI
\newcommand\doilink[1]{\href{http://dx.doi.org/#1}{#1}}
\newcommand\doi[1]{doi:\doilink{#1}}


%%%%%%%%%%%%%%%%%%%%%%%% End Helper Commands %%%%%%%%%%%%%%%%%%%%%%%%%%%

%%%%%%%%%%%%%%%%%%%%%%%%% Begin CV Document %%%%%%%%%%%%%%%%%%%%%%%%%%%%

\usepackage{graphicx}
\begin{document}


\begin{figure}%[4]
        \vspace{-2.3cm}
      \hspace{12.9cm}
		\href{https://github.com/tUrG0n/portfolio}{\includegraphics[scale=0.7]{git.png}}
		\vspace{-2.0cm}
\end{figure}


% For a left forkme png
%\begin{figure}%[4]
%    \flushleft
%		\vspace{-2.6cm}
%		\hspace{-6.0cm}
%		\href{https://github.com/tUrG0n/Resume}{\includegraphics[scale=0.7]{git.png}}
%		\vspace{-2.0cm}
%\end{figure}



\makeheading{
Piotr Yordanov \hspace{2.50cm} +1-415-612-0587 \hspace{2.50cm}piotr.yordanov@gmail.com
}

\section{Contact}
%
% NOTE: Mind where the & separators and \\ breaks are in the following
%       table.
%
% ALSO: \rcollength is the width of the right column of the table
%       (adjust it to your liking; default is 1.85in).
%
\newlength{\rcollength}\setlength{\rcollength}{1.65in}%
\begin{tabular}[t]{@{}p{0.85\rcollength}p{0.85\rcollength}p{0.75\rcollength}p{0.75\rcollength}}
\href{http://lb.linkedin.com/in/piotry}{\textit{Linkedin}} &
\href{http://piotry.me}{\textit{Website} } & 
\href{http://blog.piotry.me}{\textit{Blog}}  &
\href{http://resume.github.com/?tUrG0n}{\textit{Github r\'esum\'e} }\\
\end{tabular} 

\hspace{-3.25cm}
\rule{1.225\linewidth}{0.5pt}%
%\section{Objective}
%
%Placement in an academic faculty position doing research in artificial intelligence

%%\section{Security Clearance}
%
%%Department of Defense Top Secret SCI with polygraph (expired: 2002)


%\section{Interested in}
%\begin{center}
%	\vspace{-0.7cm}
%\textit{\textbf{Startup} and \textbf{Entrepreneurship} in: }\\
%\end{center}
%\textbf{Web Development}, \textbf{Mobile Development} and
%\textbf{Artificial Intelligence} 


\section{Objective}
After a couple of start-up experiences as a technical lead, I want to work in a company where I can have a big impact on consumers

% Skills
\section{Tech Skills}
\textbf{Web stack:} node.js, backbone.js | HTML5 and css3 | building and using REST Api's

% \textbf{AI:} Recommendation Systems, Preference Recognition, Machine-Leaning

\textbf{Other:} Mongodb, Android dev, openGL, \TeX{}, AI \& Machine Learning

\textbf{Tools:} vim, git, zsh, tmux, mc, Linux/Mac


% Profressional Experience
\section{Professional Experience}
\textbf{Lead Engineer and Designer at \href{http://pauseandshine.com/}{Pause \& Shine} \hfill \small April 2012 - June 2013} 
\begin{outerlist}
\item[] Built and \textbf{designed} the website and a \textbf{mobile optimized}  app.
\end{outerlist}

\vspace{0.2cm}
\hspace{0.25cm}\textbf{\textit{Technologies}}: 
HTML5, CSS3, nodejs, backbonejs, mongodb

\vspace{0.4cm}


\textbf{Co-founder of \href{http://baytbaytak.com}{Baytbaytak} (Trulia of the middle east) \hfill \small Sep 2012 - March 2013} 
\begin{innerlist}
\item Built a platform that helped people search for appartments on a map-based website.  
\item Setup analytics to mesure customer behavior and track conversion rates. 
\item Responsible for testing and deploying the app, and involved in strategic decisions.
\end{innerlist}

\vspace{0.2cm}
\hspace{0.25cm}\textbf{\textit{Technologies}}: 
nodejs, backbonejs, mongodb, gMaps API, Google Analytics, Mixpanel

\vspace{0.4cm}

\textbf{
  Software Engineer at \href{http://anghami.com/}{Anghami}, spotify of the middle east
  \hfill \small June 2012 - August 2013
}
\begin{innerlist}
\item Researched state of the art techniques related to recommendation systems; Content-based and Collaborative filtering.
\item Tested available recommendation system technologies (Hadoop, Google compute engine)
\item Built a recommender system using lastfm's API.
\end{innerlist}


\vspace{0.2cm}
\hspace{0.25cm}\textbf{\textit{Technologies}}: 
nodejs, mongodb, lastfm API, REST API's

\vspace{0.4cm}

\textbf{Summer Scholar at \href{http://www.ri.cmu.edu/}{Carnegie Mellon University, Robotics Institute}}

Google Core AI Project, a \textit{predictive indoor Navigation application}
\hfill \small Summer 2011 \\

\begin{innerlist}
\item Made the product available to mobile consumers by porting the technology to Android.
\item Created a server-side prediction module that reduced the computational load on the client by 30x.
\end{innerlist}

\vspace{0.2cm}
\hspace{0.25cm}\textbf{\textit{Technologies}}: 
Java, android, HMM, MDP

%\vspace{0.6cm}

%\textbf{\href{http://webfea.fea.aub.edu.lb/itunit/home.aspx}{\textbf{ IT unit}} assisant at AUB's Faculty of Engineering and Architecture} 
%\begin{outerlist}
%\item[] \textit{I provided IT support to faculty, staff, and graduate students}
%        \hfill \small Fall 2008 - Spring 2010 \normalsize
%\end{outerlist}

% Projects
%\section{Projects}
%\textbf{All recent projects:} hosted on \href{http://resume.github.com/?tUrG0n}{\textit{\textbf{Github} } }
%\blankline
%\textbf{MusAR:} A preference aware, museum Augmented Reality Android App.
%\blankline


\vspace{-0.1cm}

% Publications
\section{Publications} % (fold)
\href{http://www.ri.cmu.edu/}{\textbf{Carnegie Mellon University}} Intelligent agent lab

\begin{outerlist}
    \item[] \href{http://www.ri.cmu.edu/pub_files/2011/12/percom-navigation-main.pdf}{Predictive Indoor Navigation using Commercial Smart-phones}, In: 28th Annual ACM Symposium on Applied Computing, 2013.
\end{outerlist}
\label{sec:section}

% Education
\section{Education}
%%% AUB
\href{http://www.aub.edu.lb/}{\textbf{The American University of Beirut}},
Beirut, Lebanon
\begin{outerlist}
    \item[] B.E., \href{http://webfea.fea.aub.edu.lb/ece/} {Electrical and Computer Engineering}, Class of 2012, 3 times on Dean's Honnor List
\end{outerlist}

\blankline

%%% GLFL

\href{http://www.glfl.edu.lb/}{\textbf{Grand Lyc\'ee Franco Libanais}},
Beirut, Lebanon
\begin{outerlist}
\item[] 
    \href{http://www.education.gouv.fr/cid143/le-baccalaureat.html}{French Baccalaureate}, Mention \textbf{`Tr\'es bien'} (High Distinction), 2008
\end{outerlist}

\vspace{-0.2cm}

% Awards
\section{Awards} % (fold)
Winner of Startup Weekend Byblos \textit{March 2013}\\
Winner of Startup Weekend Beirut \textit{July 2012} 
\label{sec:section}

\vspace{-0.1cm}

% Organizations
\section{Organizations} % (fold)
Creator of the \href{https://www.facebook.com/groups/390113624411131/}{Lebanese Open Source Software Dev} community. (December 2012)\\
\begin{outerlist}
\item[] 
    
\end{outerlist}
\label{sec:section}


%%% CNSM
%\textbf{Conservatoire National Sup\'erieur de Musique Beyrouth}
%\begin{outerlist}
%\item[] Classical and Jazz curriculum, 1998 - 2012
%\end{outerlist}




%%%% Horizontal Line

%\hspace{-3.4cm}
%\rule{1.3\linewidth}{1pt}%

%%%% Horizontal Line


% Languages
%\section{languages}
%\textit{Fluent} in English, French and Arabic. \textit{Notions} of Russian and Italian

% section section (end)
 
\end{document}
%%%%%%%%%%%%%%%%%%%%%%%%%% End CV Document %%%%%%%%%%%%%%%%%%%%%%%%%%%%%
